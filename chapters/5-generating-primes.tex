\chapter{Erzeugung von Primzahlen}

Für public-key-Kryptosysteme benötigt man häufig zufällige große Primzahlen. In der Regel erzeugt man dazu eine natürliche Zahl in der gewünschten Größe und
prüft, ob diese eine Primzahl ist, beispielsweise versucht man diese Zahl $n$ durch alle Primzahlen $\leq \sqrt{n}$ zu teilen, dieses Vorgehensweise ist jedoch sehr 
ineffizient und heißt Probedivision\index{Probedivision}. Sie wird auch in Faktorisierungsalgorithmen mit Primzahlen bis $10^6$ verwendet.

\paragraph{Verteilung der Primzahlen}

Bezeichne $\pi(n)$ die Anzahl der Primzahlen im Intervall $[2, n]$. Es gilt 

$$\pi(n) \sim \frac{n}{\ln(n)} $$


\begin{center}
    \begin{tabular}{ ll } 
        \hline
        $n$ & $\pi(n)$  \\ 
        \hline
                 10 &         4 \\
                100 &        25 \\
               1000 &       168 \\
              10000 &      1229 \\
             100000 &      9592 \\
            1000000 &     78498 \\
           10000000 &    664579 \\
          100000000 &   5761455 \\
         1000000000 &  50847534 \\
        10000000000 & 455052512 \\
        \hline
    \end{tabular}
\end{center}

\paragraph{Abschätzung des Aufwands}

Aus $\pi(n) \sim n / \ln(n)$ folgt: man benötigt wenigstens $\sqrt{n} / \ln(\sqrt{n})$ Probedivisionen, um zu
beweisen, daß eine natürliche Zahl $n$ eine Primzahl ist.
Beispielsweise werden bei RSA Primzahlen verwendet, die aktuell in der Regel größer als
$10^{150}$ sind.\\

Um nun die Primalität für diese Zahl zu beweisen, müßte man insgesamt mehr als
$10^{150/2} / \ln(10^{150/2}) > 5.7 \cdot 10^{72}$ Probedivisionen durchführen -- das ist nicht durchführbar!
Somit sucht man nach effizienteren Verfahren: Primzahltests\index{Primzahltest} stellen mit einer
hohen Wahrscheinlichkeit fest, ob eine Zahl auch eine Primzahl ist (probabilistic primality
tests).
% Angewandte_Kryptographie_SoSe2025_part1_v01.pdf, Seite 34ff

\section{Der Fermat Test}
Der Fermat-Test
- der Test beruht auf dem kleinen Satz von Fermat: Ist n eine Primzahl, so gilt 

$$a^{n-1} \equiv 1  \mod n$$ 

für alle $a \in \mathbb{Z}$ mit gcd($a,n$) = 1.

Somit kann man mit diesem Satz überprüfen, ob eine Zahl zusammengesetzt ist:

\begin{enumerate}
    \item man wählt dazu eine natürliche Zahl $a \in \{1,2, \ldots, n-1\}$
    \item als nächste berechnet man $r = a^{n-1} \mod n$
    \item ist nun $r \neq 1$, so ist $n$ keine Primzahl und zusammengesetzt. Ergibt sich $r = 1$, so kann $n$ eine Primzahl oder zusammengesetzt sein
    \item die Schritte 1 bis 3 müssen für alle Werte von $a$ nun durchlaufen werden, wenn $r = 1$
\end{enumerate}

Es braucht die Festlegung einer ``Sicherheits-Grenze'': Wie oft muss der Algorithmus bei $r = 1$ durchlaufen werden, damit man sicher sein kann, eine Primzahl gefunden 
zu haben?\\
    
\noindent Der Fermat-Test kann zeigen, dass $n$ zusammengesetzt ist, er kann aber nicht beweisen, dass $n$ eine Primzahl ist. \\

\noindent Das Verfahren ist zur Faktorisierung ungeeignet.

\paragraph{Beispiel}
Gegeben: $n = 341 = 11 \cdot 31$,  d.h. $n$ ist keine Primzahl. Wir berechnen für $a = 2$

$$2^{340} \equiv 1 \mod 341,$$

obwohl $n$ zusammengesetzt ist. Für ein anderen $a$, z.B. $a = 3$ haben wir 

$$3^{340} \equiv 56 \mod 341.$$

Im Ergebnis bedeutet das, dass 341 also eine zusammengesetzte Zahl ist und somit nicht prim sein kann.

\section{Carmichael-Zahlen}

Wenn der Fermat-Test für viele Basen $a$ keine Bestätigung für ein zusammengesetztes $n$
gefunden hat, ist es wahrscheinlich, dass $n$ eine Primzahl ist.
Es existieren aber natürliche Zahlen, deren Eigenschaft ist, dass sie zusammengesetzt sind,
das jedoch nicht mit dem Fermat-Test gezeigt werden kann. \\

Sei $n$ eine ungerade zusammengesetzte Zahl und es gilt für eine ganze Zahl $a$ folgende
Kongruenz

$$a^{n-1} \equiv 1 \mod n,$$

so nennt man $n$ eine Pseudoprimzahl zur Basis $a$. Sei $n$ nun eine Pseudoprimzahl zur Basis $a$ für alle ganzen Zahlen $a$ mit gcd($a,n$) = 1. Dann
heißt $n$ Carmichael-Zahl\index{Carmichael-Zahl}. 

\paragraph{Beispiel} für eine Carmichael-Zahl: $561 = 3 \cdot 11 \cdot 17$.

\paragraph{Eigenschaften} Eine ungerade zusammengesetzte Zahl $n \geq 3$ ist eine Carmichael-Zahl, wenn genau
diese zwei Bedingungen gelten:
\begin{enumerate}
    \item $n$ ist quadratfrei, d.h. $n$ hat keinen mehrfachen Primteiler
    \item $p - 1$ teilt $n - 1$ für alle Primteiler $p$ von $n$
\end{enumerate}

Somit folgt, dass jede Carmichael-Zahl ein Produkt von mindestens drei unterschiedlichen
Primzahlen ist. Daher wissen wir auch, dass unendlich viele Carmichael-Zahlen existieren.

\section{Der Miller-Rabin Test}

Im Gegensatz zum Fermat-Test findet der Miller-Rabin-Test nach ``ausreichend'' vielen
Durchläufen für jede natürliche Zahl heraus, ob diese zusammengesetzt ist. \\


\noindent Sei $n$ eine natürliche ungerade Zahl mit $s = \max\{r \in \mathbb{N} | 2^r \text{ teilt } n-1\}$.
Damit ist $2^s$ die größte Potenz, die $n-1$ teilt oder anders ausgedrückt: es gibt ein ungerades $d \in \mathbb{Z}$, sodass $n - 1 = 2^s \cdot d$.

\begin{theorem}
    Ist $n$ eine Primzahl und $a$ eine zu $n$ teilerfremde ganze Zahl, so gilt mit den
obigen Bezeichnungen entweder
    \begin{enumerate}
        \item $a^d \equiv 1 \mod n$
        \item$a^{2^r \cdot d} \equiv -1 \mod n$
    \end{enumerate}
\end{theorem}

Mindestens eine der Bedingungen muss erfüllt sein, dass $n$ eine Primzahl ist. Weiters gilt, $n$ ist keine Primzahl (also zusammengesetzt) g.d.w. man eine ganze zu $n$ 
teilerfremde Zahl $a$ findet, für die weder die Bedingung (1) noch (2) für ein $r \in \{0, 1, \ldots, s-1\}$ gilt.
Dann wird $a$ Zeuge gegen die Primalität von $n$ genannt.

\paragraph{Beispiel}
Sei $n = 561$ und $a = 2$. Wir behaupten $a$ ist eine Zeuge gegen die Primalität von $n$:

Wir berechnen $s = 4$ und wählen $d = 35$, dann gilt

\begin{align*}
    2^{35} &\equiv 263 \mod 561 \\
    2^{2\cdot 35} &\equiv 166 \mod 561 \\ 
    2^{4\cdot 35} &\equiv 67 \mod 561 \\
    2^{8\cdot 35} &\equiv 1 \mod 561
\end{align*}

Also ist 561 keine Primzahl nach vorhergehenden Theorem.

\paragraph{Abschätzung der Anzahl der Zeugen gegen eine Primalität einer Zahl $n$}

\begin{theorem}
Sei $n \geq 3$ eine ungerade zusammengesetzte Zahl, so gibt es in der Menge
$\{1,2, \ldots, n-1\}$ höchstens $(n-1)/4$ Zahlen, die zu $n$ teilerfremd sind und keine Zeugen gegen die
Primalität von $n$ sind.
\end{theorem}

Für einen Beweis: vgl. Buchmann J., Einführung in die Kryptographie, 3. Auflage, S. 127 ff.

\paragraph{Beispiel}: Bestimmung aller Zeugen gegen eine Primalität der Zahl $n$.
Sei $n = 15$, es ist $n-1 = 14 = 2 \cdot 7$, daraus folgt: $s = 1$ und $d = 7$. Eine zu 15 teilerfremde Zahl $a$ ist genau dann Zeuge gegen die Primzahleigenschaft
von $n$ wenn $a^7 \mod 15 \neq \pm 1$ gilt.

In einer tabellarischen Übersicht:


\begin{center}
    \begin{tabular}{ ccc ccc ccc } 
        \hline
        $a$ & 1 & 2 & 4 & 7 & 8 & 11 & 13 & 14 \\ 
        \hline
        $a^{14} \mod 15$ &  1 &  4 &  1 &  4 &  4 &  1 &  4 &  1 \\
        $a^{ 7} \mod 15$ &  1 &  8 &  4 & 13 &  2 & 11 &  7 & 14 \\ 
        \hline
    \end{tabular}
\end{center}

Die Anzahl der zu 15 teilerfremden Zahlen in $\{1, 2,\dots,14\}$, die keine Zeugen gegen die
Primalität von $n = 15$ sind, beläuft sich auf $2 \leq (15 - 1)/4 = 3.5$.

\paragraph{Anwendung} des Miller-Rabin-Tests auf eine ungerade Zahl $n$

\begin{itemize}
    \item man wählt zufällig und gleichverteilt eine Zahl aus der Menge $\{2,3,\dots, n-1\}$
    \item ist der gcd($a,n$) > 1, so ist $n$ zusammengesetzt
    \item ist der gcd($a,n$) = 1, so testet man $a^d, a^{2d}, a^{2^2d}, \ldots, a^{2^{s-1}d}$
    \item findet man nun einen Zeugen gegen die Primalität von $n$, dann wurde gezeigt, dass $n$ zusammengesetzt ist
    \item die Wahrscheinlichkeit dafür, dass $n$ zusammengesetzt ist und man keinen Zeugen findet, beläuft sich nach obigen Theorem auf höchstens $0.25$.
\end{itemize}

Ist $n$ zusammengesetzt, so findet man bei der Iteration des Miller-Rabin-Tests mit $t$ Durchläufen mit einer Wahrscheinlichkeit von höchstens $frac14t$ keinen Zeugen 
gegen die Primalität von $n$.

\paragraph{Beispiel} nach 10 Durchläufen ergibt sich eine Wahrscheinlichkeit von $frac14 10$, das entspricht ungefähr einer $0.0001$-prozentigen Wahrscheinlichkeit, 
dass man keinen Zeugen gegen die Primzahleigenschaft von $n$ findet.

\section{Verfahren zur zufälligen Wahl von Primzahlen}

Beim Verfahren für die Erzeugung einer zufälligen $k$-Bit-Primzahl: dabei wird zuerst eine zufällige und ungerade $k$-Bit-Zahl $n$ generiert.
Davon wird das erste und letzte Bit wird auf 1 gesetzt, die restlichen $k-2$ Bits werden zufällig und gleichverteilt gesetzt. 
Jetzt wird überprüft, ob $n$ eine Primzahl ist:
\begin{itemize}
    \item ist $n$ durch eine Primzahl unter einer gewissen Schranke $B$ (in der Regel wird $B = 10^6$ gesetzt) teilbar? Die Primzahlen werden in einer Tabelle vorgehalten.
    \item wird dabei kein Teiler von $n$ gefunden, so wird der Miller-Rabin-Test mit $t$ Wiederholungen ( mit $t \geq 1$ als entsprechender Sicherheitsparameter) auf $n$ 
    angewendet
    \item wird dabei kein Zeuge gegen die Primzahleigenschaft von $n$ gefunden, so gilt $n$ nun als Primzahl (mit der Sicherheit $\frac{t}{4}$).
    \item ansonsten muß der gesamte Test mit einer anderen Zahl $n'$ wiederholt werden
\end{itemize}

Die Auswahl der Schranke $B$ hängt von dem Verhältnis der Ausführungszeiten einer
Probedivision und eines Miller-Rabin-Tests auf der verwendeten Hard- bzw. Software-Plattform ab.